\documentclass[12pt,-letter paper]{article}
\usepackage{siunitx}
\usepackage{setspace}
\usepackage{gensymb}
\usepackage{xcolor}
\usepackage{caption}
%\usepackage{subcaption}
\doublespacing
\singlespacing
\usepackage[none]{hyphenat}
\usepackage{amssymb}
\usepackage{relsize}
\usepackage[cmex10]{amsmath}
\usepackage{mathtools}
\usepackage{amsmath}
\usepackage{commath}
\usepackage{amsthm}
\interdisplaylinepenalty=2500
%\savesymbol{iint}
\usepackage{txfonts}
%\restoresymbol{TXF}{iint}
\usepackage{wasysym}
\usepackage{amsthm}
\usepackage{mathrsfs}
\usepackage{txfonts}
\let\vec\mathbf{}
\usepackage{stfloats}
\usepackage{float}
\usepackage{cite}
\usepackage{cases}
\usepackage{subfig}
%\usepackage{xtab}
\usepackage{longtable}
\usepackage{multirow}
%\usepackage{algorithm}
\usepackage{amssymb}
%\usepackage{algpseudocode}
\usepackage{enumitem}
\usepackage{mathtools}
%\usepackage{eenrc}
%\usepackage[framemethod=tikz]{mdframed}
\usepackage{listings}
%\usepackage{listings}
\usepackage[latin1]{inputenc}
%%\usepackage{color}{   
%%\usepackage{lscape}
\usepackage{textcomp}
\usepackage{titling}
\usepackage{hyperref}
%\usepackage{fulbigskip}   
\usepackage{tikz}
\usepackage{graphicx}
\lstset{
  frame=single,
  breaklines=true
}
\let\vec\mathbf{}
\usepackage{enumitem}
\usepackage{graphicx}
\usepackage{siunitx}
\let\vec\mathbf{}
\usepackage{enumitem}
\usepackage{graphicx}
\usepackage{enumitem}
\usepackage{tfrupee}
\usepackage{amsmath}
\usepackage{amssymb}
\usepackage{mwe} % for blindtext and example-image-a in example
\usepackage{wrapfig}
\graphicspath{{figs/}}
\providecommand{\mydet}[1]{\ensuremath{\begin{vmatrix}#1\end{vmatrix}}}
\providecommand{\myvec}[1]{\ensuremath{\begin{bmatrix}#1\end{bmatrix}}}
\providecommand{\cbrak}[1]{\ensuremath{\left\{#1\right\}}}
\providecommand{\brak}[1]{\ensuremath{\left(#1\right)}}
\begin{document}
\begin{enumerate}
	
\item Let $R$ be the set of real numbers. Determine all functions $f:R\rightarrow R$ such that, for all real numbers $z$ and $y$, \begin{align}f(f(x)f(y))+f(x+y)=f(xy)\end{align}.
\item $A$ hunter and an invisible rabbit play a game in the Euclidean plane. The rabbit's starting point, $Ag$, and the hunter's starting point, $Bo$, are the same. After $n-1$ rounds of the game, the rabbit is at point $An-$ and the hunter is at point $B-1$. In the $nth$ round of the game, three things occur in order.                             (i) The rabbit moves invisibly to a point $A$, such that the distance between $An-1$ and $A$,, is exactly $1$.
(ii) $A$ tracking device reports a point $P$, to the hunter. The only guarantee provided by the tracking device to the hunter is that the distance between $P$ and $A$, is at most $1$.
(iii) The hunter moves visibly to a point $B$, such that the distance between $Bu-1$ and $Bn$ is exactly $1$. Is it always possible, no matter how the rabbit moves, and no matter what points are reported by the tracking device, for the hunter to choose her moves so that after $10$ rounds she can ensure that the distance between her and the rabbit is at most $1002$
\begin{enumerate}[label=(\roman*)]
	\item The rabbit moves invisibly to a point An such that the distance between $An-1$ and An is exactly $1$.
	\item $A$ tracking device reports a point $Pa$ to the hunter. The only guarantee provided by the tracking device to the hunter is that the distance between $P$, and $An$, is at most $1$.
	\item The hunter moves visibly to a point $B,$ such that the distance between $B-1$ and $B$, is exactly $1$.
\end{enumerate}
Is it always possible, no matter how the rabbit moves, and no matter what points are reported by the tracking device, for the hunter to choose her moves so that after $10$ rounds she can ensure that the distance between her and the rabbit is at most $100?$
\item Let Rand $S$ be different points on a circle and such that $RS$ is not a diameter. Let $E$ be the tangent line to $2$ at $R$. Point $T$ is such that $S$ is the midpoint of the line segment $RT$. Point $J$ is chosen on the shorter are $RS$ of $Q$ so that the circumcircle $I$ of triangle $JST$ intersects ( at two distinct points. Let $A$ be the common point of $I$ and that is closer to $R$. Line $AJ$ meets again at $K$. Prove that the line $KT$ is tangent to $\gamma$.
\item An integer $N\leq2$ is given. A collection of $N(N+1)$ soccer players, no two of whom are of the same height, stand in a row. Sir Alex wants to remove $N(N-1)$ players from this row leaving a new row of $2N$ players in which the following $V$ conditions hold
	\begin{enumerate}[label=(\arabic *)]
		\item no one stands between the two tallest players,                 \item no one stands between the third and fourth tallest players
	
\item no one stands between the two shortest players.
	\end{enumerate}
	Show that this is always possible.
	
\item An ordered pair $(x, y)$ of integers is a primitive point if the greatest common divisor of $r$ and $y$ is $1$. Given a finite set $S$ of primitive points, prove that there exist a positive integer $n$ and integers $ao$, $41$, $4$ such that, for each $(x, y)4$ in $S$, we have
	\begin{align}a_0x^n + a_1x^{n-1}y + a_2x^{n-2}y^2 + \cdots + a_{n-1}xy^{n-1} + a_ny^n = 1.\end{align}




\end{enumerate}
\end{document}




