\documentclass[12pt,-letter paper]{article}
\usepackage{siunitx}
\usepackage{setspace}
\usepackage{gensymb}
\usepackage{xcolor}
\usepackage{caption}
%\usepackage{subcaption}
\doublespacing
\singlespacing
\usepackage[none]{hyphenat}
\usepackage{amssymb}
\usepackage{relsize}
\usepackage[cmex10]{amsmath}
\usepackage{mathtools}
\usepackage{amsmath}
\usepackage{commath}
\usepackage{amsthm}
\interdisplaylinepenalty=2500
%\savesymbol{iint}
\usepackage{txfonts}
%\restoresymbol{TXF}{iint}
\usepackage{wasysym}
\usepackage{amsthm}
\usepackage{mathrsfs}
\usepackage{txfonts}
\let\vec\mathbf{}
\usepackage{stfloats}
\usepackage{float}
\usepackage{cite}
\usepackage{cases}
\usepackage{subfig}
%\usepackage{xtab}
\usepackage{longtable}
\usepackage{multirow}
%\usepackage{algorithm}
\usepackage{amssymb}
%\usepackage{algpseudocode}
\usepackage{enumitem}
\usepackage{mathtools}
%\usepackage{eenrc}
%\usepackage[framemethod=tikz]{mdframed}
\usepackage{listings}
%\usepackage{listings}
\usepackage[latin1]{inputenc}
%%\usepackage{color}{   
%%\usepackage{lscape}
\usepackage{textcomp}
\usepackage{titling}
\usepackage{hyperref}
%\usepackage{fulbigskip}   
\usepackage{tikz}
\usepackage{graphicx}
\lstset{
  frame=single,
  breaklines=true
}
\let\vec\mathbf{}
\usepackage{enumitem}
\usepackage{graphicx}
\usepackage{siunitx}
\let\vec\mathbf{}
\usepackage{enumitem}
\usepackage{graphicx}
\usepackage{enumitem}
\usepackage{tfrupee}
\usepackage{amsmath}
\usepackage{amssymb}
\usepackage{mwe} % for blindtext and example-image-a in example
\usepackage{wrapfig}
\graphicspath{{figs/}}
\providecommand{\mydet}[1]{\ensuremath{\begin{vmatrix}#1\end{vmatrix}}}
\providecommand{\myvec}[1]{\ensuremath{\begin{bmatrix}#1\end{bmatrix}}}
\providecommand{\cbrak}[1]{\ensuremath{\left\{#1\right\}}}
\providecommand{\brak}[1]{\ensuremath{\left(#1\right)}}
\begin{document}
\begin{enumerate}
	\item Triangle $BCF$ has a right angle at $B$. Let $A$ be the point on line $CF$ such that \begin{align}FA=FB and F\end{align} lies between $A$ and $C$. Point $D$ is chosen such that \begin{align}DA = DC and AC\end{align} is the bisector of $\angle DAB.$ Point $E$ is chosen such that \begin{align}EA= ED and AD\end{align} is the bisector of $\angle EAC$. Let $M$ be the midpoint of $CF$. Let $X$ be the point such that $AMXE$ is a parallelogram \begin{align}(where AM || EX and AE || MX)\end{align}. Prove that lines \begin{align}BD, FX, and ME\end{align} are concurrent.
		\item Find all positive integers $n$ for which each cell of an $n\times n$ table can be filled with one of the letters \begin{align}I, M and O\end{align} in such a way that:
 in each row and each column,one third of the entries are $I$,one third are $M$ and one third are $O;$and
		in any diagonal,if the number of entries on the diagonal is a multiple of three,the$n$ one third of the entries are $I,$ one thirdv are $M$ and one third are $O.$
			\item \begin{align}Let P=A_1A_2... A_k\end{align} be a convex polygon in the plane. The vertices \begin{align}A_1, A_2,... A_k\end{align} have integral coordinates and lie on a circle. Let $S$ be the area of $P$. An odd positive integer $n$ is given such that the squares of the side lengths of $P$ are integers divisible by $n$. Prove that $2S$ is an integer divisible by $n$.
\item $A$ set of positive integers is called fragrant if it contains at least two elements and each of its elements has a prime factor in common with at least one of the other elements. Let $P(n) = n ^ 2 + n + 1$ What is the least possible value of the positive integer $b$ such that there exists a non-negative integer a for which the set

	\begin{align}{P(a + 1), P(a + 2) ,...,P(a+b)}\end{align}
is fragrant?	
\item The equation \begin{align}(x - 1)(x - 2)....(x-2016)=(x-1)(x-2)...(x-2016)\end{align} is written on the board, with $2016$ linear factors on each side. What is the least possible value of $k$ for which it is possible to erase exactly $k$ of these $4032$ linear factors so that at least one factor remains on each side and the resulting equation has no real solutions?
\item There are $n\leq 2$ line segments in the plane such that every two segments cross, and no three segments meet at a point. Geoff has to choose an end point of each segment and place a frog on it, facing the other end point. Then he will clap his hands $n-1$ times. Every time he claps. each frog will immediately jump forward to the next intersection point on its segment. Frogs never change the direction of their jumps. Geoff wishes to place the frogs in such a way that no two of them will ever occupy the same intersection point at the same time.

(a) Prove that Geoff can always fulfil his wish if $n$ is odd.

(b) Prove that Geoff can never fulfil his wish if $n$ is even.


\end{enumerate}
\end{document}
